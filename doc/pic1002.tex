\documentclass[12pt,a4paper]{jsarticle}
\usepackage[dvipdfmx]{graphicx}
\usepackage[margin=6truemm]{geometry}
\usepackage{caption}
\usepackage{booktabs}
\captionsetup[figure]{skip=0pt}
\makeatletter
\setlength{\@fptop}{3pt}
\setlength{\@fpsep}{3pt}
\setlength{\@fpbot}{0pt plus 1fil}
\makeatother
\graphicspath{{../png/}}
\begin{document}
\section{contents}
河川流量のチューニング。基本的な数値設定は以下の通り。

河川1.2倍
\begin{center}
  \begin{tabular}{lr} \toprule
    variables  & value \\ \midrule
    Horizontal\_mixing\_coefficient & 0.25 \\
    HPRNU & 0.25 \\
    VPRNU** & 0.12 \\
    Vertical\_mixing\_coefficient & 1e-7 \\ \bottomrule
  \end{tabular}
\end{center}


\section{観測点}
\begin{figure}[hbtp]
  \includegraphics[keepaspectratio,width=100mm]{map/st_location_outer_revised.png}
  \caption{地図}
\end{figure}

\section{塩分コンターマップ}
\begin{figure}
  \begin{tabular}{cc}
    \begin{minipage}[t]{0.5\hsize}
      \centering
      \includegraphics[keepaspectratio,width=90mm]{contour/salinitydynTokyo831_chiba1buoy.png}
      \caption{千葉港口第一号灯標}
    \end{minipage} 
    \begin{minipage}[t]{0.5\hsize}
      \centering
      \includegraphics[keepaspectratio,width=90mm]{contour/salinitydynTokyo831_chibaharo.png}
      \caption{千葉波浪観測塔}
    \end{minipage} \\
    \begin{minipage}[t]{0.5\hsize}
      \centering
      \includegraphics[keepaspectratio,width=90mm]{contour/salinitydynTokyo831_kawasaki.png}
      \caption{川崎}
    \end{minipage} 
    \begin{minipage}[t]{0.5\hsize}
      \centering
      \includegraphics[keepaspectratio,width=90mm]{contour/salinitydynTokyo831_urayasu.png}
      \caption{浦安}
    \end{minipage} 
  \end{tabular}
\end{figure}


\clearpage
\begin{figure}[hbtp]
    \begin{tabular}{c}
      \begin{minipage}[t]{0.5\hsize}
        \centering
        \includegraphics[keepaspectratio, width=180mm]{Tokyo831/salinity_chiba1buoy.png}
        \caption{千葉港口第一号灯標}
      \end{minipage} \\
      \begin{minipage}[t]{0.5\hsize}
        \centering
        \includegraphics[keepaspectratio, width=180mm]{Tokyo831/salinity_chibaharo.png}
        \caption{検見川沖(千葉波浪観測塔)}
      \end{minipage} \\
      \begin{minipage}[t]{0.5\hsize}
        \centering
        \includegraphics[keepaspectratio, width=180mm]{Tokyo831/salinity_urayasu.png}
        \caption{浦安}
      \end{minipage} \\
      \begin{minipage}[t]{0.5\hsize}
        \centering
        \includegraphics[keepaspectratio, width=180mm]{Tokyo831/salinity_kawasaki.png}
        \caption{川崎}
      \end{minipage} \\
    \end{tabular}
  \end{figure}


\clearpage
\section{水温変化のコンター}
\begin{figure}[hbtp]
    \includegraphics[keepaspectratio,width=100mm]{contour/tempdynTokyo831_chiba1buoy.png}
    \caption{千葉港口第一号灯標の水温変化}
\end{figure}

\begin{figure}[hbtp]
  \begin{tabular}{c}
    \begin{minipage}[t]{0.5\hsize}
      \centering
      \includegraphics[keepaspectratio, width=180mm]{Tokyo831/temperature_chiba1buoy.png}
      \caption{千葉港口第一号灯標}
    \end{minipage} \\
    \begin{minipage}[t]{0.5\hsize}
      \centering
      \includegraphics[keepaspectratio, width=180mm]{Tokyo831/temperature_chibaharo.png}
      \caption{検見川沖(千葉波浪観測塔)}
    \end{minipage} \\
    \begin{minipage}[t]{0.5\hsize}
      \centering
      \includegraphics[keepaspectratio, width=180mm]{Tokyo831/temperature_urayasu.png}
      \caption{浦安}
    \end{minipage} \\
    \begin{minipage}[t]{0.5\hsize}
      \centering
      \includegraphics[keepaspectratio, width=180mm]{Tokyo831/temperature_kawasaki.png}
      \caption{川崎}
    \end{minipage} \\
  \end{tabular}
\end{figure}




\end{document}
