\documentclass[fontsize=12pt,paper=a4]{jlreq}
\usepackage{graphicx}
%\usepackage[margin=6truemm]{geometry}
%\usepackage{caption}
\usepackage{booktabs}
\graphicspath{{../png/}}

%begin
\begin{document}
\section{Tokyo10\_run.nmlの結果 quick view}
Tokyo1シリーズは、 HMC,VMC,VPRNUを変化させたもので、
メッシュは内湾、高解像度
\begin{table}
  \begin{minipage}[hbtp]{0.5\hsize}
    \begin{tabular}{lrr} \toprule
      パラメータ & 値 \\ \midrule
      HPRNU & 0.3\\
      Horizontal\_mixing\_coefficient & 0.3\\
      VPRNU & 0.12\\
      Vertical\_mixing\_coefficient & 1e-6\\
      河川流量 & 1.4倍 \\
      風力 & 1.4倍\\ \bottomrule
    \end{tabular}
  \end{minipage}
\end{table}

%salinity
\begin{figure}[hbtp]
        \centering
        \includegraphics[keepaspectratio, width=170mm]{Tokyo10/salinity_kawasaki.png}
        \caption{川崎人工島での塩分 左から表層、中層、底層\\緑点は公共用水域、水色が実験値、オレンジが観測値}
\end{figure}

\begin{figure}[hbtp]
        \centering
        \includegraphics[keepaspectratio, width=170mm]{contour/salinitydynrunbulkTokyo10_kawasaki.png}
        \caption{川崎人工島での塩分コンターマップ}
\end{figure}


%temp
\begin{figure}[hbtp]
        \centering
        \includegraphics[keepaspectratio, width=170mm]{Tokyo10/temperature_kawasaki.png}
        \caption{川崎人工島での水温 左から表層、中層、底層\\緑点は公共用水域、水色が実験値、オレンジが観測値}
\end{figure}
\begin{figure}[hbtp]
  \centering
  \includegraphics[keepaspectratio, width=170mm]{Tokyo10/temperature_urayasu.png}
  \caption{urayasuでの水温 左から表層、中層、底層\\緑点は公共用水域、水色が実験値、オレンジが観測値}
\end{figure}

\begin{figure}[hbtp]
  \centering
  \includegraphics[keepaspectratio, width=170mm]{Tokyo10/temperature_chiba1buoy.png}
  \caption{chiba1buoyでの水温 左から表層、中層、底層\\緑点は公共用水域、水色が実験値、オレンジが観測値}
\end{figure}

\begin{figure}[hbtp]
  \centering
  \includegraphics[keepaspectratio, width=170mm]{Tokyo10/temperature_chibaharo.png}
  \caption{chibaharoでの水温 左から表層、中層、底層\\緑点は公共用水域、水色が実験値、オレンジが観測値}
\end{figure}


\begin{figure}[hbtp]
    \centering
    \includegraphics[keepaspectratio, width=170mm]{contour/tempdynrunbulkTokyo10_kawasaki.png}
    \caption{川崎人工島での水温コンターマップ}
\end{figure}


%kh
\begin{figure}[hbtp]
  \centering
  \includegraphics[keepaspectratio, width=170mm]{contour/khrunbulkTokyo10_kawasaki.png}
  \caption{川崎人工島での鉛直渦動拡散係数コンターマップ}
\end{figure}

\end{document}