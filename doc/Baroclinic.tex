\documentclass[fontsize=12pt,paper=a4]{jlreq}
\usepackage{graphicx}
\usepackage{booktabs}
\graphicspath{{../png/}}

%begin
\begin{document}
\section{tvdb\_run.nmlの結果 quick view}
バロクリニック流れを実装した.
\begin{table}
  \begin{minipage}[hbtp]{0.5\hsize}
    \begin{tabular}{lrr} \toprule
      パラメータ & 値 \\ \midrule
      HPRNU & 0.4\\
      Horizontal\_mixing\_coefficient & 0.1\\
      VPRNU & 1.0\\
      Vertical\_mixing\_coefficient & 1e-6\\
      河川流量 & 1.0倍 \\
      風力 & 1.4倍\\ \bottomrule
    \end{tabular}
  \end{minipage}
\end{table}

%salinity

%コンター
\begin{figure}[hbtp]
  \begin{tabular}{cc}
    \begin{minipage}[hbtp]{0.5\hsize}
      \includegraphics[keepaspectratio, width=90mm]{contour/salinitytvdb_chibaharo.png}
      \caption{検見川沖での塩分コンターマップ}
    \end{minipage} & 
    \begin{minipage}[hbtp]{0.5\hsize}
      \includegraphics[keepaspectratio, width=90mm]{contour/salinitytvdb_kawasaki.png}
      \caption{川崎人工島での塩分コンターマップ}
    \end{minipage} \\
    \begin{minipage}[hbtp]{0.5\hsize}
      \includegraphics[keepaspectratio, width=90mm]{contour/salinitytvdb_chiba1buoy.png}
      \caption{chiba1buoyでの塩分コンターマップ}
    \end{minipage} & 
    \begin{minipage}[hbtp]{0.5\hsize}
      \includegraphics[keepaspectratio, width=90mm]{contour/salinitytvdb_urayasu.png}
      \caption{浦安での塩分コンターマップ}
    \end{minipage} \\
  \end{tabular}
\end{figure}


\begin{figure}[hbtp]
  \begin{tabular}{cc}
    \begin{minipage}[hbtp]{0.9\hsize}
      \includegraphics[keepaspectratio, width=170mm]{tvdb/salinity_chibaharo.png}
      \caption{検見川沖での塩分コンターマップ}
    \end{minipage} \\
    \begin{minipage}[hbtp]{0.9\hsize}
      \includegraphics[keepaspectratio, width=170mm]{tvdb/salinity_kawasaki.png}
      \caption{川崎人工島での塩分コンターマップ}
    \end{minipage} \\
    \begin{minipage}[hbtp]{0.9\hsize}
      \includegraphics[keepaspectratio, width=170mm]{tvdb/salinity_chiba1buoy.png}
      \caption{chiba1buoyでの塩分コンターマップ}
    \end{minipage} \\
    \begin{minipage}[hbtp]{0.9\hsize}
      \includegraphics[keepaspectratio, width=170mm]{tvdb/salinity_urayasu.png}
      \caption{浦安での塩分コンターマップ}
    \end{minipage} \\
  \end{tabular}
\end{figure}



%temp
\begin{figure}[hbtp]
        \centering
        \includegraphics[keepaspectratio, width=170mm]{tvdb/temperature_kawasaki.png}
        \caption{川崎人工島での水温 左から表層、中層、底層\\緑点は公共用水域、水色が実験値、オレンジが観測値}
\end{figure}

\begin{figure}[hbtp]
  \centering
  \includegraphics[keepaspectratio, width=170mm]{tvdb/temperature_urayasu.png}
  \caption{浦安での水温 左から表層、中層、底層\\緑点は公共用水域、水色が実験値、オレンジが観測値}
\end{figure}

\begin{figure}[hbtp]
  \centering
  \includegraphics[keepaspectratio, width=170mm]{tvdb/temperature_chiba1buoy.png}
  \caption{chiba1buoyでの水温 左から表層、中層、底層\\緑点は公共用水域、水色が実験値、オレンジが観測値}
\end{figure}

\begin{figure}[hbtp]
  \centering
  \includegraphics[keepaspectratio, width=170mm]{tvdb/temperature_chibaharo.png}
  \caption{検見川沖での水温 左から表層、中層、底層\\緑点は公共用水域、水色が実験値、オレンジが観測値}
\end{figure}


\begin{figure}[hbtp]
    \centering
    \includegraphics[keepaspectratio, width=170mm]{contour/temptvdb_chibaharo.png}
    \caption{検見川沖での水温コンターマップ}
\end{figure}


%kh
%\begin{figure}[hbtp]
%  \centering
%  \includegraphics[keepaspectratio, width=170mm]{contour/khrunesttvdb_kawasaki.png}
%  \caption{川崎での鉛直渦動拡散係数コンターマップ}
%\end{figure}

\end{document}