\documentclass[report,fontsize=12pt]{ltjsarticle}
\usepackage{graphicx}
\graphicspath{{../png/}}
\begin{document}

\section{VPRNUの変化による結果}
VPRNU=1,3で結果を確認した.
湾中央付近として川崎人工島,湾奥として検見川沖(千葉波浪塔)を採用する.

VPRNU =3のときの図を図(\ref{VPRNU=1}~\ref{VPRNU=1おわり})に示す.
\begin{figure}[hbtp]
    \centering
    \includegraphics[width=140mm]{tvdb_v3/temperature_chibaharo.png}
    \caption{検見川沖水温}
    \label{VPRNU=1}
\end{figure}
\begin{figure}[hbtp]
    \centering
    \includegraphics[width=140mm]{tvdb_v3/temperature_kawasaki.png}
    \caption{川崎水温}
\end{figure}
\begin{figure}[hbtp]
    \centering
    \includegraphics[width=140mm]{tvdb_v3/salinity_chibaharo.png}
    \caption{検見川沖塩分}
\end{figure}
\begin{figure}[hbtp]
    \centering
    \includegraphics[width=140mm]{tvdb_v3/salinity_kawasaki.png}
    \caption{川崎塩分}
    \label{VPRNU=1おわり}
\end{figure}

\clearpage
同時にプロットしたものがこちら
\begin{figure}[hbtp]
    \centering
    \includegraphics[width=100mm]{cross_plot/upper_salinity_kawasakib3.png}
    \caption{川崎塩分-表層}

\end{figure}
\begin{figure}[hbtp]
    \centering
    \includegraphics[width=100mm]{cross_plot/bottom_salinity_kawasakib3.png}
    \caption{川崎塩分-底層}

\end{figure}
\clearpage
\begin{table}
    \centering
    \begin{tabular}[b]{|l||r|} \hline
      HMC   & 0.4 \\
      HPRNU & 0.1 \\
      VMC & 1e-6 \\
      風力 & 1.4倍 \\ \hline

     \end{tabular}
\caption{考察を通じて変化させなかった値}
\end{table}

\end{document}