\documentclass[fontsize=12pt,paper=a4]{jlreq}
\usepackage{graphicx}
%\usepackage[margin=6truemm]{geometry}
%\usepackage{caption}
\usepackage{booktabs}
%\captionsetup[figure]{skip=0pt}
%\makeatletter
%\setlength{\@fptop}{3pt}
%\setlength{\@fpsep}{3pt}
%\setlength{\@fpbot}{0pt plus 1fil}
%\makeatother
\graphicspath{{../png/}}
\begin{document}
水温が低い問題を受けて、メッシュの領域を変化させることでどんな影響があるか調べた。(河川1.8倍,風力1.4倍*Tokyo610のみ風力1.4倍,河川1.0倍)


\begin{table}
  \begin{minipage}[hbtp]{0.5\hsize}
    \begin{tabular}{lrr} \toprule
      パラメータ & 値 \\ \midrule
      HPRNU & 0.25\\
      Horizontal\_mixing\_coefficient & 0.25\\
      VPRNU & 0.12\\
      Vertical\_mixing\_coefficient & 1e-7\\
      風力 & 1.4倍\\
      河川流量 & 1.8倍\\ \bottomrule
    \end{tabular}
  \end{minipage}
  \begin{minipage}[hbtp]{0.45\hsize}
    \begin{tabular}{lrr} \toprule
      label & メッシュの種類 \\ \midrule
      Tokyo610 & 内湾、高解像度\\
      Tokyo921 & 外洋、低解像度\\
      Tokyo1001 & 内湾、低解像度\\ \bottomrule
    \end{tabular}
  \end{minipage}
\end{table}


\section{salinity}
説明:同色が同じシミュレーションケースでの表層と底層を表しており、黒線は観測データ。
\begin{figure}[hbtp]
        \centering
        \includegraphics[keepaspectratio, width=170mm]{cross_plot/salinity_kawasaki71.png}
        \caption{川崎人工島での塩分}
  \end{figure}



\section{temperature}
  \begin{figure}[hbtp]
        \centering
        \includegraphics[keepaspectratio, width=170mm]{cross_plot/temperature_kawasaki71.png}
        \caption{川崎人工島での水温}
  
  \end{figure}


\end{document}