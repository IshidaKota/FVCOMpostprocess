\documentclass[fontsize=12pt,paper=a4]{jlreq}
\usepackage{graphicx}
%\usepackage[margin=6truemm]{geometry}
%\usepackage{caption}
\usepackage{booktabs}
\graphicspath{{../png/}}

%begin
\begin{document}
\section{Tokyo922\_run.nmlの結果 quick view}
河川流量を変化させる実験において、流量2.0倍のケースである。

\begin{table}
  \begin{minipage}[hbtp]{0.5\hsize}
    \begin{tabular}{lrr} \toprule
      パラメータ & 値 \\ \midrule
      HPRNU & 0.25\\
      Horizontal\_mixing\_coefficient & 0.25\\
      VPRNU & 0.12\\
      Vertical\_mixing\_coefficient & 1e-7\\
      風力 & 1.4倍\\ \bottomrule
    \end{tabular}
  \end{minipage}
\end{table}

%salinity
\begin{figure}[hbtp]
        \centering
        \includegraphics[keepaspectratio, width=170mm]{Tokyo922/salinity_kawasaki.png}
        \caption{川崎人工島での塩分 左から表層、中層、底層\\緑点は公共用水域、水色が実験値、オレンジが観測値}
\end{figure}

\begin{figure}[hbtp]
        \centering
        \includegraphics[keepaspectratio, width=170mm]{contour/salinitydynTokyo922_kawasaki.png}
        \caption{川崎人工島での塩分コンターマップ}
\end{figure}


%temp
\begin{figure}[hbtp]
        \centering
        \includegraphics[keepaspectratio, width=170mm]{Tokyo922/temperature_kawasaki.png}
        \caption{川崎人工島での水温 左から表層、中層、底層\\緑点は公共用水域、水色が実験値、オレンジが観測値}
\end{figure}

\begin{figure}[hbtp]
    \centering
    \includegraphics[keepaspectratio, width=170mm]{contour/tempdynTokyo922_kawasaki.png}
    \caption{川崎人工島での水温コンターマップ}
\end{figure}

\end{document}